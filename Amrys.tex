\documentclass{article}
\usepackage{fontspec}
\directlua{
  fonts.handlers.otf.addfeature{
    name = "nolg",
    type = "multiple",
    data = {
      ['c_h'] = { "c", "h" },
      ['c_t'] = { "c", "t" },
      ['s_p'] = { "s", "p" },
      ['s_t'] = { "s", "t" },
      ['f_i'] = { "f", "i" },
      ['f_l'] = { "f", "l" },
    },
  }
  fonts.handlers.otf.addfeature{
    name = "rlig",
    type = "ligature",
    data = {
      ['f_f_i'] = { "f", "f", "i" },
      ['f_f_l'] = { "f", "f", "l" },
      ['f_i'] = { "f", "i" },
      ['f_l'] = { "f", "l" },
    },
  }
}
\setmainfont{Amrys}[
  Contextuals=Alternate,
  Ligatures=Rare,
  Numbers=OldStyle,
  RawFeature=+nolg]
\begin{document}
Amrys has ‘ffi’ and ‘ffl’ ligatures which don’t appear unless one
turns on discretionary ligatures (this suffices for the italic) or
breaks the standard ‘fi’ and ‘fl’ ligatures, restores them as required
ligatures, and turns on discretionary ligatures (this fixes both
italic and roman).

I’ve also broken the ‘ch’, ‘ct’, ‘sp’, and ‘st’ ligatures, since they
aren’t appropriate in every document where ‘ffi’ and ‘ffl’ ligatures
are welcome.
\end{document}
