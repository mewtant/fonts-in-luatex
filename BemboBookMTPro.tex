\documentclass{article}
\usepackage{fontspec,realscripts}
\directlua{
  fonts.handlers.otf.addfeature{
    name = "sups",
    type = "substitution",
    data = {
      four = "four.numerator",
      five = "five.numerator",
      six = "six.numerator",
      seven = "seven.numerator",
      eight = "eight.numerator",
      nine = "nine.numerator",
      zero = "zero.numerator",
    },
  }
  fonts.handlers.otf.addfeature{
    name = "bigq",
    type = "alternate",
    data = {
      Q = "Q.001",
    },
  }
  fonts.handlers.otf.addfeature{
    name = "medq",
    type = "alternate",
    data = {
      Q = "Q.002",
    },
  }
}
\setmainfont{Bembo Book MT Pro}[
  Numbers={OldStyle,Proportional},
  Style=Alternate]
\renewcommand\footnotemarkfont{\addfontfeature{RawFeature=-onum;-pnum}}
\usepackage[hidelinks]{hyperref}
\begin{document}
Bembo Book MT Pro has only three superior figures , and its ‘sups’
feature defines them to replace only the default figures. As long as
you need only footnotes numbered no higher than three, the
‘realscripts’ package will work if you turn off ‘onum’ and ‘pnum’ for
footnote marks, as above. Unfortunately, the fuller sets of superiors
in Cardo and Poliphilus MT Pro are not suitable for use with Bembo
Book. But since numerators are visually indistinguishable from
superiors in this font, ‘directlua’ can press them into service when
you need both Bembo Book and notes numbered beyond three.

‘salt’ shortens the leg of R\@. As Paul Shaw points
out,\footnote{\href{http://www.paulshawletterdesign.com/2011/05/flawed-typefaces/}{www.paulshawletterdesign.com/2011/05/flawed-typefaces/}}
the default R should be an alternate used only in titling. So turn on
‘salt’ for the document as a whole, and turn it off in titling, drop
caps, etc.

The default Q has a short tail;
{\addfontfeatures{RawFeature=+bigq}Q.001} has a very long tail; and
{\addfontfeatures{RawFeature=+medq}Q.002} has a longish tail. The
‘calt’ feature controls the tail of Q, but unintelligently (see, for
instance, this spectacular collision:
{\addfontfeatures{Contextuals=Alternate}QJ).}  Re-writing ‘calt’
(under a different name, so that the new version needn’t fight the
old) would be more trouble than it’s worth; better to leave ‘calt’
off, to decide which Q to use by default depending on the language and
the purpose of a document, and to add features like those above,
allowing for easy use of another Q when the context demands.
\end{document}
