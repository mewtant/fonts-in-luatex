\documentclass{article}
\usepackage{fontspec}
\directlua{
  fonts.handlers.otf.addfeature{
    name = "rlig",
    type = "ligature",
    data = {
      ['f_f_i'] = { "f", "f", "i" },
      ['f_f_j'] = { "f", "f", "j" },
      ['f_f_l'] = { "f", "f", "l" },
      ['f_b'] = { "f", "b" },
      ['f_f'] = { "f", "f" },
      ['f_h'] = { "f", "h" },
      ['f_i'] = { "f", "i" },
      ['f_j'] = { "f", "j" },
      ['f_k'] = { "f", "k" },
      ['f_l'] = { "f", "l" },
      ['f_t'] = { "f", "t" },
      ['k_k'] = { "k", "k" },
      ['l_l'] = { "l", "l" },
      ['t_t'] = { "t", "t" },
      ['z_z'] = { "z", "z" },
      ['Q_u'] = { "Q", "u" },
      ['T_h'] = { "T", "h" },
    },
  }
}
\setmainfont{dollyproregular}[
  UprightFeatures={Contextuals=Alternate},
  ItalicFont={*italic},
  ItalicFeatures={Contextuals=Alternate},
  SmallCapsFont={*smallcaps},
  Ligatures=NoCommon]
\begin{document}
Dolly Pro’s ligatures work perfectly in xetex, but in luatex, the
‘ffi,’ ‘ffl,’ ‘ft,’ ‘tt,’ and ‘ll’ ligatures sometimes work and
sometimes break, even within a single file.

I’ve been able to fix this only by turning off ‘liga’ and restoring
the ligatures as ‘rlig,’ paying close attention to the order in which
ligatures are defined.

Localization adds complexities I haven’t figured out, as explained in
the comments.

% If writing Catalan, add these lines to rlig:
%       ['ldot'] = { "l", "periodcentered" },
%       ['Ldot'] = { "L", "periodcentered" },


% If writing Dutch, add these lines to rlig (the first two at the top):
%       ['f_f_ij'] = { "f", "f", "i", "j" },
%       ['f_ij'] = { "f", "i", "j" },
%       ['ij'] = { "i", "j" },
%       ['IJ'] = { "I", "J" },
%
% But this will give you Dutch ligatures where they’re not
% appropriate, as in FIJI and BIJOU.  Dolly’s original features remove
% them, but I don’t know how to do a localized, contextual removal of
% ligatures with \directlua{...}.
\end{document}
